\chapter{Implementation and testing}
\label{cha:implandtest}

%---------------------------------------------------------------------------

\section{Hardware setting}
\label{sec:setting}

\usetikzlibrary{patterns}
\usetikzlibrary{calc}
\usetikzlibrary{decorations.pathmorphing}

\newcommand{\nvar}[2]{%
    \newlength{#1}
    \setlength{#1}{#2}
}

% Define a few constants for drawing
\nvar{\dg}{0.3cm}
\def\dw{0.25}\def\dh{1.3}
\nvar{\ddx}{1.5cm}

% Define commands for links, joints and such
\def\link{\draw [double distance=1.5mm, thick] (0,0)--}
\def\joint{%
    \filldraw [fill=white] (0,0) circle (5pt);
    \fill[black] circle (2pt);
}
\def\grip{%
	
    \filldraw [fill=white, thick] (0cm,\dg) rectangle (3mm,-\dg);    
    
    %\draw[ultra thick](0cm,\dg)--(0cm,-\dg)--(3mm,-\dg)--(3mm,\dg) -- cycle;
    \fill (1.5mm, -\dg) -- +(-1.5mm,-1.5mm) -- +(1.5mm,-1.5mm);
    %\fill (0cm, -0.5\dg)+(0cm,1.5pt) -- +(0.6\dg,0cm) -- +(0pt,-1.5pt);
}
\def\robotbase{%
    \draw[thick](-0.5,0) rectangle (0.5,-0.3);
    \draw[thick](-1,-0.3) -- (0.6,-0.3)  -- (0.85,-0.65) -- (0.6,-1) -- (-1,-1);
	\draw[decorate, decoration={snake, segment length=0.43cm,amplitude=.9}] (-1,-0.3)  -- (-1,-1); 
    \filldraw [fill=white, thick] (0.35,-0.9) circle (0.4);    
    
    \draw (-1.5,-\dh)-- (1.5,-\dh);
    \fill[pattern=north east lines] (-1.5,-\dh) rectangle (1.5,-\dh-0.5);
}

% Draw an angle annotation
% Input:
%   #1 Offset (optional)
%   #2 Angle
%   #3 Label
% Example:
%   \angann{30}{$\theta_1$}
\newcommand{\angann}[3][\ddx]{%
    \begin{scope}[red]
    \draw [dashed, red] (0,0) -- (1.2*#1,0pt);
    \draw [->, shorten >=3.5pt] (#1,0pt) arc (0:#2:#1);
    % Unfortunately automatic node placement on an arc is not supported yet.
    % We therefore have to compute an appropriate coordinate ourselves.
    \node at (#2/2-2:#1+8pt) {#3};
    \end{scope}
}

% Draw line annotation
% Input:
%   #1 Line offset (optional)
%   #2 Line angle
%   #3 Line length
%   #5 Line label
% Example:
%   \lineann[1]{30}{2}{$L_1$}
\newcommand{\lineann}[4][0.5]{%
    \begin{scope}[rotate=#2, blue,inner sep=2pt]
        \draw[dashed, blue!40] (0,0) -- +(0,#1)
            node [coordinate, near end] (a) {};
        \draw[dashed, blue!40] (#3,0) -- +(0,#1)
            node [coordinate, near end] (b) {};
        \draw[|<->|] (a) -- node[fill=white] {#4} (b);
    \end{scope}
}

% Draw view
% Input:
%   #1 Line offset (optional)
%   #2 Line angle
%   #3 Line length
%   #5 Line label
% Example:
%   \lineann[1]{30}{2}{$L_1$}
\newcommand{\view}[4]{%
    \begin{scope}[rotate=#2, blue,inner sep=2pt]
        \draw[dashed, blue!40] (0,0) -- +(0,#1)
            node [coordinate, near end] (a) {};
        \draw[dashed, blue!40] (#3,0) -- +(0,#1)
            node [coordinate, near end] (b) {};
        \draw[|<->|] (a) -- node[fill=white] {#4} (b);
    \end{scope}
}


% Define the kinematic parameters of the three link manipulator.
\def\thetaone{120}
\def\Lone{2}
\def\thetatwo{-60}
\def\Ltwo{2}
\def\thetathree{-20}
\def\Lthree{1}

\begin{tikzpicture}
    \lineann[1.5]{-90}{\dh}{$L_0$}
    \robotbase
    \angann[1.2cm]{\thetaone}{$\theta_1$}
    \lineann[0.7]{\thetaone}{\Lone}{$L_1$}
    \link(\thetaone:\Lone);
    \joint
    \begin{scope}[shift=(\thetaone:\Lone), rotate=\thetaone]
        \angann{\thetatwo}{$\theta_2$}
        \lineann[-0.5]{\thetatwo}{\Ltwo}{$L_2$}
        \link(\thetatwo:\Ltwo);
        \joint
        \begin{scope}[shift=(\thetatwo:\Ltwo), rotate=\thetatwo]
            \angann{\thetathree}{$\theta_3$}
            \lineann[0.7]{\thetathree}{\Lthree}{$L_3$}
            \draw [dashed, red,rotate=\thetathree] (0,0) -- (1.2\ddx,0pt);
            \link(\thetathree:\Lthree);
            \joint
            \begin{scope}[shift=(\thetathree:\Lthree), rotate=\thetathree]
                \grip
                \begin{scope}[shift={(1.5mm,-\dg)}, rotate=-90]
                \draw (0,0) -- (7,0);
	            \end{scope}
            \end{scope}
        \end{scope}
    \end{scope}
\end{tikzpicture}


%---------------------------------------------------------------------------

\section{Algorithm implementation}
\label{sec:implementatnion}


%---------------------------------------------------------------------------

\section{Testing environment and results}
\label{sec:testing}


\chapter*{Introduction}
\markboth{Introduction}{Introduction}
\label{cha:introduction}
\addcontentsline{toc}{chapter}{Introduction}  

%---------------------------------------------------------------------------


The aim of this work is to design and implement a mobile manipulation robotic platform with a vision system for obstacle detection and object recognition with a three dimensional camera. This work is a part of a project named "Mobile set of manipulators on a wheeled chassis". The project was realized in a three person team, as part of the Second Edition of ABB Students Scientific Association programme organized by ABB Corporate Research Center in Cracow.

	In the first chapter, a design of the robotic system is presented. General concept of mobile manipulation is described and possible application areas are mentioned. Subsequently, design requirements, selected hardware components, and implemented software architecture is described. The second chapter provides information about modern depth map acquisition techniques, including stereo, time-of-flight and structured light cameras. For each method, its operation principle basics and general advantages and disadvantages are provided. The third section is focused on modern methods of depth image analysis. Concepts of point clouds and point descriptors are introduced. Moreover, general object-recognition pipeline is provided.
	
	The final section presents an implementation of selected object recognition method. Test results in the real environment are provided. Possible future improvements to the algorithm are also mentioned.
	
\begin{comment}

Tematem pracy jest opis wybranych elementów składowych projektu o nazwie: Mobilny zespół manipulatorów na wspólnej platformie jezdnej. Projekt ten zrealizowany został w ramach Drugiej Edycji Koła Naukowego ABB we współpracy z Korporacyjnym Centrum Badawczym ABB w Krakowie. Zespół projektowy składał się z trzech osób pomiędzy które zostały podzielone zadania. Na tej podstawie zrealizowane zostały trzy prace dyplomowe inżynierskie. Niniejsza, związana z projektem mechanicznym, samodzielnym montażem wszystkich podzespołów oraz z analizą zagadnień kinematycznych zespołu dwóch skonstruowanych manipulatorów oraz dwie inne prace związane z oprogramowaniem operatorskim oraz z systemem akwizycji i przetwarzania danych. Projekt był realizowany w okresie od marca do listopada 2014 roku i jego główną częścią było uruchomienie zbudowanego urządzenia i jego prezentacja podczas uroczystego podsumowania.

\end{comment}










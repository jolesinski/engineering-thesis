\chapter{Analysis of the depth data}
\label{cha:analysis}

This chapter presents the main tools used further in the implementation of the autonomous control mode for the MMS robot. 

A point cloud is a set of data points in some coordinate system. In a three-dimensional coordinate system, these points are usually defined by X, Y, and Z coordinates, and often are intended to represent the external surface of an object.

Point clouds may be created by 3D scanners. These devices measure a large number of points on an object's surface, and often output a point cloud as a data file. The point cloud represents the set of points that the device has measured.



%---------------------------------------------------------------------------

\section{Basic point cloud processing}
\label{sec:pointclouds}

View transformations, filtering, outlier removal, surface normals computation.

%---------------------------------------------------------------------------

\section{Random Sample Consensus algorithm}
\label{sec:ransac}

Ransac applications (plane, cylinder) and basics. Pros and cons. Extensions. 

%---------------------------------------------------------------------------

\section{Descriptors for object recognition}
\label{sec:descriptors}


%---------------------------------------------------------------------------

\section{Iterative Closest Point algorithm}
\label{sec:icp}

%---------------------------------------------------------------------------

\section{Object recognition pipeline}
\label{sec:pipeline}

\begin{figure}[H]
\begin{center}
\begin{tikzpicture}
  [node distance = 5mm,auto,every node/.style={rectangle,draw,align=center, font=\footnotesize}]
  
  \node (kpextr) at (1,10) {Key Point \\ Extraction};
  \node[right=of kpextr] (descr1) {Description};
  \node[right=of descr1] (match1) {Matching};
  \node[draw=none,fill=none, node distance=2mm, above=of match1](ann1) {\small Recognition Pipeline for Local Descriptors};
  \node[right=of match1] (corr) {Correspondence \\ Grouping};
  \node[right=of corr] (absor) {Absolute \\ Orientation};
  \node[below right=of absor] (icp) {ICP \\ Refinement};
  \node[right=of icp] (verify) {Hypothesis \\ Verification};
  \node[below left=of icp] (align) {Alignment};
  \node[left=of align] (match2) {Matching};
  \node[left=of match2] (descr2) {Desription};
  \node[left=of descr2] (segm) {Segmentation};
  \node[draw=none,fill=none, node distance=2cm, below=of ann1](ann2) {\small Recognition Pipeline for Global Descriptors};


  \foreach \from/\to in {kpextr/descr1,descr1/match1, match1/corr, corr/absor, absor/icp, icp/verify, segm/descr2, descr2/match2, match2/align, align/icp}
    \draw[->] (\from) -- (\to);

\end{tikzpicture}

\caption{Object recognition pipeline}

\label{fig:objpipe}

\end{center}
\end{figure}
